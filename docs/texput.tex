% Emacs, this is -*-latex-*-

\title{\href{https://github.com/Tecnologias-Multimedia/intercom}{Real
    Time Audio Intercommunicator  (RTAI)}}

\author{Vicente González Ruiz}

% \makeglossaries
\maketitle
\tableofcontents
\section{The
  \href{https://python-sounddevice.readthedocs.io}{\texttt{sounddevice}
    module}}
%{{{

\texttt{sounddevice} is a \href{https://www.python.org}{Python}
\href{https://docs.python.org/3/tutorial/modules.html}{module}
available for Linux, OSX and MS Windows. \texttt{sounddevice} is a
Python wrapper for the \href{http://www.portaudio.com}{PortAudio}
library, which allows us to handle
\href{https://en.wikipedia.org/wiki/Pulse-code_modulation}{PCM}
\href{https://en.wikipedia.org/wiki/Digital_audio}{audio} in the
previous platforms.

%}}}

\section{The
  \href{https://python-sounddevice.readthedocs.io/en/0.3.15/api/streams.html}{\texttt{sounddevice.Stream}
    object}}
%{{{

The \texttt{sounddevice.Stream} object allows to simulteneous input
and output PCM digital audio through \href{https://numpy.org}{NumPy}
arrays. The following parameters are available:
\begin{enumerate}
\item \texttt{samplerate}: The desired sampling frequency (for both,
  input and output) in frames per second. By default,
  \texttt{samplerate}=44100.
\item \texttt{blocksize}: Number of frames (single samples
  in the case of mono audio or tuples of samples in the case of multichannel
  audio) passed to the callback function (see below). By default,
  \texttt{blocksize}=0, which means that the block size possiblely
  will have a variable size, depending on the host workload and the
  requested latency setting (see below).
\item \texttt{device}: Input and output device(s). By default, \texttt{device}=?
\item (int
\end{enumerate}

%}}}

